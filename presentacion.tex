\documentclass[xcolor=dvipsnames,slidestop,compress,mathserif, 11pt]{beamer}
\usecolortheme[named=Blue]{structure}
\usetheme[height=2mm]{Rochester}
\setbeamerfont{structure}{shape=\itshape}
\mode<presentation> {
\usetheme{Darmstadt} % Tema seleccionado.
\usecolortheme{default}%{albatross} % Color del tema.
\setbeamercovered{transparent} % Transparencia.
}
%\setbeamercovered{transparent}
%}%%%% packages y comandos personales %%%%
\setbeamertemplate{caption}[numbered]
\usepackage{ragged2e}
\usepackage{comment}
\renewcommand{\figurename}{Figura}
\renewcommand{\tablename}{Tabla}
%\usepackage{latexsym} % S��mbolos
\usepackage{amsmath}
\usepackage{array}
\usepackage{amssymb}
\usepackage{wasysym}
\usepackage{stmaryrd}
\usepackage{wrapfig}
\usepackage{multicol}
\usepackage{dsfont,float}
\usepackage{soul}
\usepackage{upgreek}
\usepackage{accents}
\usepackage{physics}
%\usepackage[lite]{mtpro2}
\usepackage{tikz}
\usepackage{lipsum}
\newcommand\Fontvi{\fontsize{6}{7.2}\selectfont}
%\usepackage{turnstile}
\font\shi=cmssdc10 scaled 700
\title[Tesis profesional]{Evoluci\'on de una funci\'on de Wigner de un amplificador param\'etrico}
\author{TESIS PROFESIONAL\\
Carlos Eduardo González Anguiano}
\institute{Departamento de Física ESFM-IPN}
\date{1 de junio de 2024}
\usepackage{graphicx}
%\graphicspath{%
%
%    {E:/Efi1-ggg/}% inserted by PCTeX
%    {/}% inserted by PCTeX
%}
\usepackage[utf8]{inputenc} % Usar latin1 causa error para acentos
\begin{document}


\maketitle


\begin{frame}
	\frametitle{Índice}
	\tableofcontents[pausesections]
\end{frame}

\section{Introducción}

\begin{frame}[c]
	\frametitle{Motivación}
	Max Planck y la \textit{catástrofe ultravioleta}
	Densidad espectral de energía: Energía por unidad de volumen de ondas electromagnéticas de frecuencia $\nu$.
	\begin{equation}
		u(T)=\int_{0}^{\infty}\rho(\nu, T)d\nu.
	\end{equation}
	La densidad de cuerpo negro predicha por termodinámica clásica difiere de datos experimentales.
	Planck propone que los estados de energía de los osciladores son discretos
	\begin{equation}
		E_n = nh\nu.
	\end{equation}
	De la energía media de los osciladores y la cuantización, se obtiene la distribución de Planck
	\begin{equation}
		\rho(\nu, T) = \frac{\hbar \nu^3}{\pi^2 c^3} \frac{1}{e^{\hbar\nu/kT}-1}.
	\end{equation}
\end{frame}


\end{document}






